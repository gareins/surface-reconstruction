\documentclass[11pt]{article}

\usepackage[utf8x]{inputenc}
\usepackage[slovene]{babel}

\usepackage[pdftex]{graphicx} % za slike

\title{\textbf{Surface Reconstruction}}
\author{O\v zbolt Menegatti\\
		Anej Placer\\
		Jurij Slabanja}
\date{}
\begin{document}

\maketitle

\section{Opis problema}

Cilj projekta je bil rekonstruirati povr"sino iz oblaka to"ck s pomo"cjo topolo"skih konceptov kot so "Cechov in Vietoris-Rips kompleks ter alfa oblike. Potrebno je bilo vizualizirati rezultate in izra"cunati nekatere topolo"ske invariante kot so homologija in Eulerjeva karakteristika. V nadaljevanju tega poglavja na kratko opi"semo te topolo"ske koncepte.

\subsection{"Cech kompleks}
Izmed implementiranih metod je "Cechov kompleks najbolj osnoven. "Ce imamo oblak to"ck, si lahko predstavljamo, da ga naredimo tako, da okoli vsake to"cke o"crtamo navidezno kroglo s polmerom $\delta$. Na podlagi teh krogel to"cke pove"zemo v simplekse in sicer tako, da pove"zemo to"cke katerih krogle imajo neni"celni presek. Na primer "ce imamo dve krogli, ki se sekata, potem moramo njuni sredi"s"ci (njuni to"cki v oblaku to"ck) povezati v 1-simplex oz. rob. "Ce imamo tri krogle, ki se sekajo ena z drugo vendar ne vse hkrati, dobimo prazen trikotnik. "Ce je presek vseh treh krogel neni"celen dobimo polni trikotnik itd. 

Problem s "Cechovim kompleksom je, da je potrebno zelo pazljivo izbrati $\delta$ parameter, saj lahko zelo hitro dobimo visoko dimenzijske simplekse, kar ni vedno za"zeljeno. Do istega problema pride, "ce oblak to"ck ni homogeno porazdeljen, ampak so to"cke ponekod bolj gosto posejane. V takih delih oblaka tudi zelo hitro dobimo visoko dimenzijske simplekse.

\subsection{Vietoris-Rips}
Vietoris-Rips kompleks je v osnovi zelo podoben "Cechovemu kompleksu, vendar tu ne operiramo ve"c z navideznimi kroglami ampak z radaljami med pari to"ck. Za Vietoris-Ripsa velja, da je nek simpleks del kompleksa, "ce je premer simpleksa manjsi ali enak $2\delta$. Razliko s "Cechom se najla"zje vidi, "ce vzamemo za oblak to"ck ogli"s"ca enakostrani"cnega trikotnika in za $\delta$ parameter pribli"zno polovico dol"zine stranice, kot je prikazano na Sliki~\ref{fig:vrdiag}. S "Cechovim kompleksom dobimo samo prazni trkotnik, medtem ko z Vietoris-Ripsom dobimo poln 2-simpleks.

\begin{figure}[!htb]
    \centering
    \includegraphics[width=0.4\textwidth]{vrdiag.png}
    \caption{Z Vietoris-Ripsom dobimo polni trikotnik, za razliko od "Cechovega kompleksa, kjer dobimo praznega}
    \label{fig:vrdiag}
\end{figure}

Izka"ze se, da si lai"cno lahko Vietoris-Ripsa razlagamo kot "Cechov kompleks, pri katerem so dodani simpleksi, ki imajo vsa svoja lica vklju"cena v "Cechovem kompleksu. Iz tega sledi, da je "Cechov kompleks podmno"zica Vietoris-Ripsovega kompleksa. Seveda naj opomnimo, da morata pri tem imeti Vietoris-Rips in "Cechov kompleks enak parameter $\delta$. 

Prav tako kot "Cechov kompleks ima tudi Vietoris-Rips probleme z visoko dimenzijskimi simpleksi.

\subsection{Alpha shapes}

\subsection{Homologija}

Bistvo algebraične topologije so homološke grupe. Topologijo predstavimo z Bettijevimi števili, ki predstavljajo rang homoloških grup. Homologija stopnje 0 predstavlja povezanost podatkov (koliko posameznih povezanih komponent imamo), homologija stopnje 1 detektira luknje in tunele, homologija stopnje 2 votline, itd. V tej nalogi smo se omejili na te glavne tri grupe. 

Povezanost diskretnih točk predstavimo s kompleksi. Povezanost je odvisna od določenih parametrov, npr. radij okoli točk, ki pove s katerimi drugimi točkami so le-te povezane. Ideja je podobna pri vseh treh obravnavanih kompleksih. Iz tega sledi, da je povezanost nekih diskretnih točk odvisna od izbranih parametrov in univerzalnega odgovora ni. 

Ta problem rešujemo s \emph{persistenco}. Ideja persistence je, da se z večanjem radija okoli točk spreminjajo kompleksi in homološke grupe. To pa nam pove katera topologija skozi spremembe \emph{persistira}.

Iz slike~\ref{homo} je razvidno, da z večanjem radija  postaja graf bolj povezan. Črte v prikazanem diagramu predstavljajo ``življensko dobo'' povezane komponente. Pri neki delti (epsilon) število presekanih črt predstavlja bettijevo število za določeno stopnjo.

\begin{figure}[htb]
    \centering
    \includegraphics[width=0.8\textwidth]{homo.png}
    \caption{Persistenca in barkoda.}
    \label{homo}
\end{figure}

Če pogledamo drugi primer je prečrtanih 6 povezanih komponent in 2 luknji. V seminarski nalogi smo prav tako za posamezen kompleks zgenerirali zgoraj opisano persistenčno tabelo, kjer se filtrira ven vse vnose (začetek/konec), ki vsebujejo delto (torej smo presekali črto).

\subsection{Euler}

\section{Pristop}

Uporabili smo knji"zico Dyonisus za izra"cun topologij in knji"zico QT za izris in nadzor aplikacije. Implementirali smo ve"c pristopov, Vietoris-Rips kompleks, Alfa oblike ter "Cechov kompleks. Izkazalo se je, da je "Cech izjemno po"casen, tako da je bil kanseje odstranjen, "se vedno pa ga lahko najdete v zakomentiranem delu kode.

Za prikaz imamo ve"c opcij, prosojnost pogleda, izris povezav in izbor oblaka to"ck. Za potrebe naloge smo implementirali tudi mo"znost izbire parametra $\delta$ ter procenta izbranih to"ck na kompleksu.

\section{Te"zave}

Te"zave smo imeli z ve"c detajli, a izpostavimo dva:

\begin{enumerate}
\item "Cechove metode nam ni uspelo z omejenim pomnilnikom izvesti v normalnem "casu.
\item Alpha oblike so druga"cne, kot na vajah. Namre"c v strukturi data ne najdemo podatkov o dol"zini povezave, da bi lahko uspe"sno filtrirali za dan delta. Problem smo rocno zaob"sli, vendar re"sitev ne dela optimalno.
\end{enumerate}

\section{Rezultati}

Iz slik v nadaljevanju si lahko pogledate delovanje programa. Program je prilo"zen v zip datoteki.

\begin{figure}[htb]
    \centering
    \includegraphics[width=0.6\textwidth]{vr_long.png}
    \caption{Izra"cun Vietoris-Ripsa s premajhnim delta}
    \label{fig:vr1}
\end{figure}


\begin{figure}[htb]
    \centering
    \includegraphics[width=0.6\textwidth]{vr_full_9639823.png}
    \caption{Izra"cun Vietoris-Ripsa z zadostnim delta, vendar malo to"ckami. "St. to"ck: 9639823}
    \label{fig:vr2}
\end{figure}

\begin{figure}[htb]
    \centering
    \includegraphics[width=0.6\textwidth]{vr_1hole.png}
    \caption{Izra"cun Vietoris-Ripsa n krogli: ena luknja}
    \label{fig:vr3}
\end{figure}

\begin{figure}[htb]
    \centering
    \includegraphics[width=0.6\textwidth]{vr_2hole.png}
    \caption{Izra"cun Vietoris-Ripsa n krogli: dve lunkji}
    \label{fig:vr4}
\end{figure}

\begin{figure}[htb]
    \centering
    \includegraphics[width=0.6\textwidth]{alpha_full.png}
    \caption{Izra"cun $\alpha$-oblik: polno, ena lunkja na dnu zajca}
    \label{fig:a1}
\end{figure}

\begin{figure}[htb]
    \centering
    \includegraphics[width=0.6\textwidth]{alpha_lowdelta.png}
    \caption{Izra"cun $\alpha$-oblik: polno, Izra"cun na vseh to"ckah, a s premajhnim $\delta$}
    \label{fig:a1}
\end{figure}

\begin{figure}[htb]
    \centering
    \includegraphics[width=\textwidth]{lines.png}
    \caption{Prikaz povezav}
    \label{fig:edges}
\end{figure}

\section{Delitev dela}

\end{document}
